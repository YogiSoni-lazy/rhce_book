
\begin{flushleft}

	\begin{itemize}
		\item The \textbf{I}nternet \textbf{C}orporation for \textbf{A}ssigned \textbf{N}ames and \textbf{N}umbers (ICANN) is the ultimate authority for domain-name assignments. 
		\bigskip
		\bigskip
		\item \textbf{Buying a domain name}
		\begin{itemize}
			\item First check if domain you want is available here or not: https://lookup.icann.org/en
			\item If it is available, you cannot "buy a domain name".
			\item You pay for the right to use a domain name for one or more years.
			\item You can renew your right after it's duration is expired.
			\item In simple words, you rent a second-level domain name.
		\end{itemize}
		\bigskip
		\bigskip
		 \item \textbf{How many subdomains are possible for domain name?}
		 \begin{itemize}
		 	\item Owner of domain can create unlimited number of subdomains.
		 	 \item Eg: utexas.edu can have second-level domain like:
		 	\begin{itemize}
		 		\item gslis.utexas.edu
		 		\item computerstore.utexas.edu
		 	\end{itemize}
		 \end{itemize}
		\bigskip
		\bigskip
		 \item \textbf{How big can be a domain name?}
		 \begin{itemize}
		 	\item Each subdomain can have its own subdomains. 
		 	\item Eg: example.subdomains.in.the.domain.utexas.edu.
			\item There can be \textbf{maximum 127 subdomains} in a single domain name. 
			\item Maximum size of each subdomain is \textbf{63} characters.
		 \end{itemize}  
		
		

		
	\end{itemize}








\end{flushleft}
\newpage