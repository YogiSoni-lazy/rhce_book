\setlength{\columnsep}{3pt}
\begin{flushleft}
	\bigskip
	\begin{itemize}
		\item Every machine on internet is identified by a numerical address.
		\item It will be very difficult to memorize the numerical address of all the machines in the network. 
		\item To solve the problem, historically, every machine in the network used \textbf{/etc/hosts} file where the name to address mapping was done.
		\item Structure of \textbf{/etc/hosts} file:
		\begin{tcolorbox}[breakable,notitle,boxrule=1pt,colback=pink,colframe=pink]
			\color{black}
			\fontdimen2\font=1em
			\newline
			\fontdimen2\font=4em
			IP-address      FQDN
			\fontdimen2\font=4pt
		\end{tcolorbox}	
	
		Eg:
		
		\begin{tcolorbox}[breakable,notitle,boxrule=-0pt,colback=black,colframe=black]
			\color{green}
			\fontdimen2\font=1em
			\# cat /etc/hosts
			\newline
			\color{white}
			192.168.2.4   server.example.com.
			\newline
			192.168.2.6	  client.example.com.
			\fontdimen2\font=4pt
		\end{tcolorbox}
		
	\end{itemize}
\end{flushleft}

\newpage





