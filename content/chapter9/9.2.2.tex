\setlength{\columnsep}{3pt}
\begin{flushleft}
	
		\bigskip
	
		\textbf{Physical Volumes (PVs) Commands:}
		
		\begin{itemize}
			\item 		\textbf{How to create physical volumes (PVs)?}
			
			\textbf{pvcreate}: Initialize physical volume(s) for use by LVM
			\begin{tcolorbox}[breakable,notitle,boxrule=-0pt,colback=pink,colframe=pink]
				\color{black}
				\fontdimen2\font=1em
				Syntax: pvcreate device\_name
				\fontdimen2\font=4pt
			\end{tcolorbox}
			
			
			Eg: Create physical volume of HDD \textbf{/dev/sda}:
			\begin{tcolorbox}[breakable,notitle,boxrule=-0pt,colback=black,colframe=black]
				\color{green}
				\fontdimen2\font=1em
				\# pvcreate /dev/sda
				\fontdimen2\font=4pt
			\end{tcolorbox}
			
			Eg: Create physical volume of HDD \textbf{/dev/sdb9}:
			\begin{tcolorbox}[breakable,notitle,boxrule=-0pt,colback=black,colframe=black]
				\color{green}
				\fontdimen2\font=1em
				\# pvcreate /dev/sdb9
				\fontdimen2\font=4pt
			\end{tcolorbox}
			
		\end{itemize}
	
	
	\bigskip
	\bigskip
	\begin{itemize}
		\item 	\textbf{How to display all physical volumes (PVs)?}
		
		\textbf{pvdisplay}: Display various attributes of physical volume(s)
		\begin{tcolorbox}[breakable,notitle,boxrule=-0pt,colback=pink,colframe=pink]
			\color{black}
			\fontdimen2\font=1em
			Syntax: pvdisplay [device\_name]
			\newline
			Syntax: pvs [device\_name]
			\fontdimen2\font=4pt
		\end{tcolorbox}
		
		Eg: Display all the available physical volumes:
		\begin{tcolorbox}[breakable,notitle,boxrule=-0pt,colback=black,colframe=black]
			\color{green}
			\fontdimen2\font=1em
			\# pvdisplay
			\fontdimen2\font=4pt
		\end{tcolorbox}
		Eg: Display a specific physical volume:
		\begin{tcolorbox}[breakable,notitle,boxrule=-0pt,colback=black,colframe=black]
			\color{green}
			\fontdimen2\font=1em
			\# pvdisplay /dev/sda1
			\fontdimen2\font=4pt
		\end{tcolorbox}
	
		\bigskip
		\bigskip
		\item  \textbf{How to delete a physical volumes (PVs)?}
		
		\textbf{pvremove}: Delete a physical volume(s).
		\begin{tcolorbox}[breakable,notitle,boxrule=-0pt,colback=pink,colframe=pink]
			\color{black}
			\fontdimen2\font=1em
			Syntax: pvremove device\_name
			\fontdimen2\font=4pt
		\end{tcolorbox}
		
		Eg: Remove \textbf{/dev/sda1} physical volumes:
		\begin{tcolorbox}[breakable,notitle,boxrule=-0pt,colback=black,colframe=black]
			\color{green}
			\fontdimen2\font=1em
			\# pvremove /dev/sda1
			\fontdimen2\font=4pt
		\end{tcolorbox}
		
	\end{itemize}
	
	\newpage
	
	\textbf{Volume Group (VGs) Commands}
	
	\begin{itemize}
		\item \textbf{How to create new volume group (VGs)?}
			
			\textbf{vgcreate}: Create a volume group.
			\begin{tcolorbox}[breakable,notitle,boxrule=-0pt,colback=pink,colframe=pink]
				\color{black}
				\fontdimen2\font=1em
				Syntax: vgcreate VG\_new PV
				\fontdimen2\font=4pt
			\end{tcolorbox}
			
					
			Eg: Create volume group named \textbf{vg0} of PV \textbf{/dev/sda}:
			\begin{tcolorbox}[breakable,notitle,boxrule=-0pt,colback=black,colframe=black]
				\color{green}
				\fontdimen2\font=1em
				\# vgcreate vg0 /dev/sda
				\fontdimen2\font=4pt
			\end{tcolorbox}
			
			Eg: Create volume group named \textbf{vg0} of PV \textbf{/dev/sdb9} and \textbf{/dev/sdc}:
			\begin{tcolorbox}[breakable,notitle,boxrule=-0pt,colback=black,colframe=black]
				\color{green}
				\fontdimen2\font=1em
				\# vgcreate vg0 /dev/sdb9 /dev/sdc
				\fontdimen2\font=4pt
			\end{tcolorbox}
			
	\end{itemize}

	\bigskip
		\begin{itemize}
		
		\item 	\textbf{How to extend an existing volumes groups (VGs)?}
		
		\textbf{vgextend}: Add one or more initialized PVs to an existing VG to extend it in size
		\begin{tcolorbox}[breakable,notitle,boxrule=-0pt,colback=pink,colframe=pink]
			\color{black}
			\fontdimen2\font=1em
			Syntax: vgextend [VG\_name] [PV\_name]
			\fontdimen2\font=4pt
		\end{tcolorbox}
		
		Eg: Extend VG named \textbf{vg0} by adding PV \textbf{/dev/sdb2}:
		\begin{tcolorbox}[breakable,notitle,boxrule=-0pt,colback=black,colframe=black]
			\color{green}
			\fontdimen2\font=1em
			\# vgextend vg0  /dev/sdb2
			\fontdimen2\font=4pt
		\end{tcolorbox}
		
		\bigskip
		\bigskip


		\item 	\textbf{How to remove a PV from an existing volumes groups (VGs)?}

		\textbf{vgreduce}: Remove a PV from a VG.
		\begin{tcolorbox}[breakable,notitle,boxrule=-0pt,colback=pink,colframe=pink]
			\color{black}
			\fontdimen2\font=1em
			Syntax: vgreduce [VG\_name] [PV\_name]
			\fontdimen2\font=4pt
		\end{tcolorbox}
		
		Eg: Remove PV named \textbf{/dev/sdb2} from VG named \textbf{vg0}
		\begin{tcolorbox}[breakable,notitle,boxrule=-0pt,colback=black,colframe=black]
			\color{green}
			\fontdimen2\font=1em
			\# vgreduce vg0  /dev/sdb2
			\fontdimen2\font=4pt
		\end{tcolorbox}
		
		\newpage	
		\item 	\textbf{How to display all volumes groups (VGs)?}
		
		\textbf{vgdisplay}: Display various attributes of volume group(s).
		\begin{tcolorbox}[breakable,notitle,boxrule=-0pt,colback=pink,colframe=pink]
			\color{black}
			\fontdimen2\font=1em
			Syntax: vgdisplay [VG\_name]
			\newline
			Syntax: vgs [VG\_name]
			\fontdimen2\font=4pt
		\end{tcolorbox}
		 	
		Eg: Display all the available volume groups:
		\begin{tcolorbox}[breakable,notitle,boxrule=-0pt,colback=black,colframe=black]
			\color{green}
			\fontdimen2\font=1em
			\# vgdisplay
			\fontdimen2\font=4pt
		\end{tcolorbox}
		Eg: Display a specific volume group:
		\begin{tcolorbox}[breakable,notitle,boxrule=-0pt,colback=black,colframe=black]
			\color{green}
			\fontdimen2\font=1em
			\# vgdisplay vg0
			\fontdimen2\font=4pt
		\end{tcolorbox}
		
		\bigskip
		\bigskip
		\item  \textbf{How to delete a volume group (VGs)?}
		
		\textbf{vgremove}: Remove a specific volume group.
		\begin{tcolorbox}[breakable,notitle,boxrule=-0pt,colback=pink,colframe=pink]
			\color{black}
			\fontdimen2\font=1em
			Syntax: vgremove VG\_name
			\fontdimen2\font=4pt
		\end{tcolorbox}
		
		Eg: Remove \textbf{vg0} volume group:
		\begin{tcolorbox}[breakable,notitle,boxrule=-0pt,colback=black,colframe=black]
			\color{green}
			\fontdimen2\font=1em
			\# vgremove vg0
			\fontdimen2\font=4pt
		\end{tcolorbox}
		
	\end{itemize}
	
	\newpage
	
	\textbf{Logical Volume (LV) Commands}
	
	\begin{itemize}
		\item \textbf{How to create new logical volume (LV)?}
		
		\textbf{lvcreate}: Create a new logical volume
		\begin{tcolorbox}[breakable,notitle,boxrule=-0pt,colback=pink,colframe=pink]
			\color{black}
			\fontdimen2\font=1em
			Syntax: lvcreate -L size[G,M,K] -n vg\_name  lv\_name
			\fontdimen2\font=4pt
		\end{tcolorbox}
		
		
		
		\bigskip
		
		Eg: Create logical volume named \textbf{lv1} from volume group \textbf{vg0} of size 50GB.
		\bigskip
		\begin{tcolorbox}[breakable,notitle,boxrule=-0pt,colback=black,colframe=black]
			\color{green}
			\fontdimen2\font=1em
			\# lvcreate -L 50G -n vg0 lv1
			\fontdimen2\font=4pt
		\end{tcolorbox}
				
	\end{itemize}
	

	\bigskip
	\begin{itemize}
	
	\item 	\textbf{How to display logical volumes (LVs)?}
	\newline
	\textbf{lvdisplay}: Check for the LV properties
	\begin{tcolorbox}[breakable,notitle,boxrule=-0pt,colback=pink,colframe=pink]
		\color{black}
		\fontdimen2\font=1em
		Syntax: lvdisplay [LV\_name]
		\newline
		Syntax: lvs [LV\_name]
		\fontdimen2\font=4pt
	\end{tcolorbox}
	
	Eg: Display all the available logical volumes:
	\begin{tcolorbox}[breakable,notitle,boxrule=-0pt,colback=black,colframe=black]
		\color{green}
		\fontdimen2\font=1em
		\# lvdisplay
		\fontdimen2\font=4pt
	\end{tcolorbox}
	Eg: Display a specific logical volume:
	\begin{tcolorbox}[breakable,notitle,boxrule=-0pt,colback=black,colframe=black]
		\color{green}
		\fontdimen2\font=1em
		\# lvdisplay /dev/vg0/lv1
		\newline
		or
		\newline
		\# lvdisplay /dev/maper/vg0-lv1
		\fontdimen2\font=4pt
	\end{tcolorbox}
	
	\bigskip
	\bigskip	
	
	\item \textbf{Once LV partition is ready, you can give it filesystem like normal partition}.
	\newline
	Eg: Apply \textbf{xfs} filesystem to LV named \textbf{/dev/mapper/vg0-lv1}
	\begin{tcolorbox}[breakable,notitle,boxrule=-0pt,colback=black,colframe=black]
	\color{green}
	\fontdimen2\font=1em
	\# mkfs -t xfs /dev/mapper/vg0-lv1
	\fontdimen2\font=4pt
	\end{tcolorbox}
	
	\bigskip
	\bigskip
	
	\item \textbf{Mount the LV partition like normal partition}.
	\newline
	Eg: Temporary mounting
	\begin{tcolorbox}[breakable,notitle,boxrule=-0pt,colback=black,colframe=black]
		\color{green}
		\fontdimen2\font=1em
		\# mount /dev/mapper/vg0-lv1 /mnt
		\fontdimen2\font=4pt
	\end{tcolorbox}

	Eg: Permanent mounting
	\begin{tcolorbox}[breakable,notitle,boxrule=-0pt,colback=black,colframe=black]
		\color{green}
		\fontdimen2\font=1em
		\# tail -1 /etc/fstab
		\color{white}
		\newline
		/dev/mapper/vg0-lv1  /mnt xfs defaults 0 0
		\fontdimen2\font=4pt
	\end{tcolorbox}
\end{itemize}	
	\bigskip
	\bigskip
	\begin{itemize}

	\item 	\textbf{How to delete an existing logical volume (LVs)?}
	
	\textbf{lvremove}: Removes one or more logical volumes.
	\begin{tcolorbox}[breakable,notitle,boxrule=-0pt,colback=pink,colframe=pink]
		\color{black}
		\fontdimen2\font=1em
		Syntax: lvremove [LV\_name]
		\fontdimen2\font=4pt
	\end{tcolorbox}
	
	Eg: Remove logical volume \textbf{/dev/mapper/vg0-lv1}
	\begin{tcolorbox}[breakable,notitle,boxrule=-0pt,colback=black,colframe=black]
		\color{green}
		\fontdimen2\font=1em
		\# lvremove /dev/mapper/vg0-lv1
		\fontdimen2\font=4pt
	\end{tcolorbox}
	\bigskip
	\begin{tcolorbox}[breakable,notitle,boxrule=-0pt,colback=red,colframe=red]
		\color{white}
		Warning: Make sure to unmount the LV partition before deleting it.
	\end{tcolorbox}
	
	
	\bigskip
	\bigskip
	\item  \textbf{How to extend an existing logical volume (LVs)?}

	\begin{itemize}
		\item Step 1:
		\newline
			\textbf{lvextend}: Extend the logical volume(s)
			\newline
			To give the final size to logical volume:
			\begin{tcolorbox}[breakable,notitle,boxrule=-0pt,colback=pink,colframe=pink]
				\color{black}
				\fontdimen2\font=1em
				Syntax: lvextend -L size[G,M,K] logical\_volume\_name
				\fontdimen2\font=4pt
			\end{tcolorbox}
			
			To give the additional size to logical volume:
			\begin{tcolorbox}[breakable,notitle,boxrule=-0pt,colback=pink,colframe=pink]
				\color{black}
				\fontdimen2\font=0.8em
				Syntax: lvextend -L +size[G,M,K] logical\_volume\_name
				\fontdimen2\font=4pt
			\end{tcolorbox}
			
			Eg: Extend logical volume with total size of 8GB:
		
			\begin{tcolorbox}[breakable,notitle,boxrule=-0pt,colback=black,colframe=black]
				\color{green}
				\fontdimen2\font=1em
				\# lvextend -L 8G /dev/test-volume/data
				\fontdimen2\font=4pt
			\end{tcolorbox}
		
			Eg: Provide additional 5GB size to logical volume:
			\bigskip
			\begin{tcolorbox}[breakable,notitle,boxrule=-0pt,colback=black,colframe=black]
				\color{green}
				\fontdimen2\font=1em
				\# lvextend -L +5G /dev/test-volume/data
				\fontdimen2\font=4pt
			\end{tcolorbox}
			
			\bigskip
			\begin{tcolorbox}[breakable,notitle,boxrule=-0pt,colback=yellow,colframe=yellow]
				\color{black}
				Note: No need to unmount the LV partition while extending it.
			\end{tcolorbox}
			
			
			\item Step 2:
			\newline
			Once the LV partition is extended, it is important to resize the filesystem.
			\begin{itemize}
				\item To extend \textbf{ext2, ext3, ext4} filesystems:
				\newline
				\textbf{resize2fs}: This command is used to resize ext, ext2, ext3 and ext4 filesystems
					\begin{tcolorbox}[breakable,notitle,boxrule=-0pt,colback=pink,colframe=pink]
						\color{black}
						\fontdimen2\font=0.8em
						Syntax: resize2fs logical\_volume\_name
						\fontdimen2\font=4pt
					\end{tcolorbox}
					
					Eg: To resize filesystem of logical volume:
					\begin{tcolorbox}[breakable,notitle,boxrule=-0pt,colback=black,colframe=black]
						\color{green}
						\fontdimen2\font=1em
						\# resize2fs /dev/mapper/vg0-lv1
						\fontdimen2\font=4pt
					\end{tcolorbox}

			\item To extend \textbf{xfs} filesystem:
			\newline
			\textbf{xfs\_growfs}: Expands an existing XFS filesystem.
			\begin{tcolorbox}[breakable,notitle,boxrule=-0pt,colback=pink,colframe=pink]
				\color{black}
				\fontdimen2\font=0.8em
				Syntax: xfs\_growfs logical\_volume\_name
				\fontdimen2\font=4pt
			\end{tcolorbox}
			
			Eg: To resize filesystem of logical volume:
			\begin{tcolorbox}[breakable,notitle,boxrule=-0pt,colback=black,colframe=black]
				\color{green}
				\fontdimen2\font=1em
				\# xfs\_growfs /dev/mapper/vg0-lv1
				\fontdimen2\font=4pt
			\end{tcolorbox}
			
			\end{itemize}
	\end{itemize}				
	
	\bigskip
	\bigskip
	
	\item  \textbf{How to reduce an existing logical volume (LVs)?}
	
	\begin{tcolorbox}[breakable,notitle,boxrule=-0pt,colback=yellow,colframe=yellow]
		\color{black}
		Note: You cannot reduce \textbf{xfs} filesystem. You can reduce \textbf{ext2, ext3 \& ext4} filesystems.
	\end{tcolorbox}
	
	\begin{itemize}
		\item Step 1: 
		\newline
		\textbf{Unmount} the LV partition. 
		\newline Eg:
		\begin{tcolorbox}[breakable,notitle,boxrule=-0pt,colback=black,colframe=black]
			\color{green}
			\fontdimen2\font=1em
			\# umount /dev/mapper/vg0-lv1
			\fontdimen2\font=4pt
		\end{tcolorbox}
		\item Step 2:
		\newline
		\textbf{e2fsck}: Used to check the ext2/ext3/ext4 family of file systems for any errors.
		The \textbf{-f} option is used to force the checking.
		\begin{tcolorbox}[breakable,notitle,boxrule=-0pt,colback=pink,colframe=pink]
			\color{black}
			\fontdimen2\font=0.8em
			Syntax: e2fsck -f logical\_volume\_name
			\fontdimen2\font=4pt
		\end{tcolorbox}
		
		Eg: Check the filesystem of logical volume:
		\begin{tcolorbox}[breakable,notitle,boxrule=-0pt,colback=black,colframe=black]
			\color{green}
			\fontdimen2\font=1em
			\# e2fsck -f /dev/mapper/vg0-lv1
			\fontdimen2\font=4pt
		\end{tcolorbox}
		
		\item Step 3:
		\newline
		\textbf{resize2fs}: Allows you to resize ext2, ext3, or ext4 file systems 
		\begin{tcolorbox}[breakable,notitle,boxrule=-0pt,colback=pink,colframe=pink]
			\color{black}
			\fontdimen2\font=0.8em
			Syntax: resize2fs logical\_volume\_name [desired size]
			\fontdimen2\font=4pt
		\end{tcolorbox}
		
		Eg: Resize the filesystem of LV:
		\begin{tcolorbox}[breakable,notitle,boxrule=-0pt,colback=black,colframe=black]
			\color{green}
			\fontdimen2\font=1em
			\# resize2fs /dev/mapper/vg0-lv1 3G
			\fontdimen2\font=4pt
		\end{tcolorbox}
		
		\item Step 4:
		\newline
		\textbf{lvreduce}: Shrink the LV and the filesystem applied on it
		\begin{tcolorbox}[breakable,notitle,boxrule=-0pt,colback=pink,colframe=pink]
			\color{black}
			\fontdimen2\font=0.8em
			Syntax: lvreduce -L size[G,M,K] logical\_volume\_name
			\fontdimen2\font=4pt
		\end{tcolorbox}
		
		Eg: Resize the filesystem of logical volume:
		\begin{tcolorbox}[breakable,notitle,boxrule=-0pt,colback=black,colframe=black]
			\color{green}
			\fontdimen2\font=1em
			\# lvreduce -L 3G /dev/mapper/vg0-lv1
			\fontdimen2\font=4pt
		\end{tcolorbox}
		
		\item Step 5:
		\newline
		Finally mount the filesystem again:
		\begin{tcolorbox}[breakable,notitle,boxrule=-0pt,colback=black,colframe=black]
			\color{green}
			\fontdimen2\font=1em
			\# mount /dev/mapper/vg0-lv1  /mnt
			\fontdimen2\font=4pt
		\end{tcolorbox}
		
	\end{itemize}

	
	\end{itemize}

	
\end{flushleft}
\newpage
	