\setlength{\columnsep}{3pt}
\begin{flushleft}
	Let's see some commands to archives and compress file or directory.
	\begin{itemize}
		\item \textbf{zip \& unzip} command: 
		\begin{enumerate}
			\item \textbf{zip} command: Compression and file/directory packaging utility. Available on Linux, Windows, and Mac OS. 
			\begin{tcolorbox}[breakable,notitle,boxrule=0pt,colback=pink,colframe=pink]
				\color{black}
				\fontdimen2\font=1em
				Syntax: zip argument *
				\fontdimen2\font=4pt
			\end{tcolorbox}
			Eg:
			\begin{tcolorbox}[breakable,notitle,boxrule=-0pt,colback=black,colframe=black]
				\color{green}
				\fontdimen2\font=1em
				\$ mkdir testiing
				\newline
				\$ cd testing
				\newline
				\$ touch {1..10} \& mkdir -p a/b/c
				\newline
				\color{white}
				\# To compress \textbf{testing} folder, execute \textbf{zip} command inside the folder:
				\newline
				\color{green}
				\$ zip data *
				\fontdimen2\font=4pt
			\end{tcolorbox}
			
			\bigskip
			\begin{tcolorbox}[breakable,notitle,boxrule=0pt,colback=yellow,colframe=yellow]
				\color{black}
				Note: No need to add \textbf{.zip} extension. Creates \textbf{data.zip} file automatically.
			\end{tcolorbox}
			
			Options for \textbf{zip} command:
			\begin{itemize}
				\item \textbf{-r}: To zip up an entire directory (including all subdirectories):
				\begin{tcolorbox}[breakable,notitle,boxrule=0pt,colback=pink,colframe=pink]
					\color{black}
					\fontdimen2\font=1em
					Syntax: zip -r argument *
					\fontdimen2\font=4pt
				\end{tcolorbox}
				Eg:
				\begin{tcolorbox}[breakable,notitle,boxrule=-0pt,colback=black,colframe=black]
					\color{green}
					\fontdimen2\font=1em
					\$ zip -r data *
					\fontdimen2\font=4pt
				\end{tcolorbox}
					
			\end{itemize}
		
		
		\item \textbf{unzip} command is used to decompress files/directories.
		\begin{tcolorbox}[breakable,notitle,boxrule=0pt,colback=pink,colframe=pink]
			\color{black}
			\fontdimen2\font=1em
			Syntax: unzip argument
			\fontdimen2\font=4pt
		\end{tcolorbox}
		Eg:
		\bigskip
		\begin{tcolorbox}[breakable,notitle,boxrule=-0pt,colback=black,colframe=black]
			\color{green}
			\fontdimen2\font=1em
			\$ unzip data
			\fontdimen2\font=4pt
		\end{tcolorbox}
		
		\end{enumerate}
	\newpage
	\item \textbf{gzip} \& \textbf{gunzip} command:
	\begin{enumerate}
		\item \textbf{gzip} command: \textbf{Can compress only files.} Available on Linux and Mac OS. Has \textbf{".gz"} file extension.
		\bigskip
		\begin{tcolorbox}[breakable,notitle,boxrule=0pt,colback=pink,colframe=pink]
			\color{black}
			\fontdimen2\font=1em
			Syntax: gzip filename
			\fontdimen2\font=4pt
		\end{tcolorbox}
		Eg:
		\begin{tcolorbox}[breakable,notitle,boxrule=-0pt,colback=black,colframe=black]
			\color{green}
			\fontdimen2\font=1em
			\$ gzip file.txt
			\fontdimen2\font=4pt
		\end{tcolorbox}
	
		\bigskip
		\begin{tcolorbox}[breakable,notitle,boxrule=0pt,colback=yellow,colframe=yellow]
			\color{black}
			Note: This command replaces \textbf{file.txt} with \textbf{file.txt.gz}
		\end{tcolorbox}
			
		
		\item \textbf{gunzip} command: Uncompress the files with \textbf{.gz} extension
		\bigskip
		\begin{tcolorbox}[breakable,notitle,boxrule=0pt,colback=pink,colframe=pink]
			\color{black}
			\fontdimen2\font=1em
			Syntax: gunzip filename.gz
			\fontdimen2\font=4pt
		\end{tcolorbox}
		Eg:
		\begin{tcolorbox}[breakable,notitle,boxrule=-0pt,colback=black,colframe=black]
			\color{green}
			\fontdimen2\font=1em
			\$ gunzip file.txt.gz
			\fontdimen2\font=4pt
		\end{tcolorbox}		
	\end{enumerate}
		
		

	\item \textbf{bzip2} \& \textbf{bunzip2} command:
	\begin{enumerate}
		\item \textbf{bzip2} command: \textbf{Can compress only files.} Available on Linux and Mac OS. Has \textbf{".bz2"} file extension.
		\bigskip
		\begin{tcolorbox}[breakable,notitle,boxrule=0pt,colback=pink,colframe=pink]
			\color{black}
			\fontdimen2\font=1em
			Syntax: bzip2 filename
			\fontdimen2\font=4pt
		\end{tcolorbox}
		Eg:
		\begin{tcolorbox}[breakable,notitle,boxrule=-0pt,colback=black,colframe=black]
			\color{green}
			\fontdimen2\font=1em
			\$ bzip2 file.txt
			\fontdimen2\font=4pt
		\end{tcolorbox}
		
		\bigskip
		\begin{tcolorbox}[breakable,notitle,boxrule=0pt,colback=yellow,colframe=yellow]
			\color{black}
			Note: This command replaces \textbf{file.txt} with \textbf{file.txt.bz2}
		\end{tcolorbox}
		
		
		\item \textbf{bunzip2} command: Uncompress the files with \textbf{.bz2} extension
		\bigskip
		\begin{tcolorbox}[breakable,notitle,boxrule=0pt,colback=pink,colframe=pink]
			\color{black}
			\fontdimen2\font=1em
			Syntax: bunzip2 filename.bz2
			\fontdimen2\font=4pt
		\end{tcolorbox}
		Eg:
		\begin{tcolorbox}[breakable,notitle,boxrule=-0pt,colback=black,colframe=black]
			\color{green}
			\fontdimen2\font=1em
			\$ bunzip2 file.txt.bz2
			\fontdimen2\font=4pt
		\end{tcolorbox}		
	\end{enumerate}		

	\newpage

	\item \textbf{tar} command: Stands for "tape archive". \textbf{Creates an archive of a directory.}
	\bigskip
	\begin{tcolorbox}[breakable,notitle,boxrule=0pt,colback=pink,colframe=pink]
		\color{black}
		\fontdimen2\font=1em
		Syntax: tar [options] archive-file-name.tar directory\_name
		\fontdimen2\font=4pt
	\end{tcolorbox}
	Options of \textbf{tar} command:
	\begin{enumerate}
		\item 
		\textbf{-c}: Create archive
		\newline
		\textbf{-v}: Stands for verbose. Tells tar to print all the filenames added to archive.
		\newline
		\textbf{-f}: The name of the archive.
		\newline
		Eg:
		\begin{tcolorbox}[breakable,notitle,boxrule=-0pt,colback=black,colframe=black]
			\color{green}
			\fontdimen2\font=1em
			\$ tar -cvf MyProject.tar  MyProject
			\fontdimen2\font=4pt
		\end{tcolorbox}		
		\bigskip
		\begin{tcolorbox}[breakable,notitle,boxrule=0pt,colback=yellow,colframe=yellow]
			\color{black}
			Note: The \textbf{-cvf} will only archive the directory, but not compress it. To compress it further, you will need to use \textbf{"gzip MyProject.tar"} command. 
		\end{tcolorbox}
		\item \textbf{-z}: Compress a tar archive with the gzip command. You need to add extension \textbf{".tar.gz"}.
		\newline
		Eg:
		\begin{tcolorbox}[breakable,notitle,boxrule=-0pt,colback=black,colframe=black]
			\color{green}
			\fontdimen2\font=1em
			\$ tar -cvzf MyProject.tar  MyProject
			\fontdimen2\font=4pt
		\end{tcolorbox}		
		\item \textbf{-j}: Compress a tar archive with bzip2 command. You need to add extension \textbf{".tar.bz2"}.
		\newline
		Eg:
		\begin{tcolorbox}[breakable,notitle,boxrule=-0pt,colback=black,colframe=black]
			\color{green}
			\fontdimen2\font=1em
			\$ tar -cjvf MyProject.tar.bz2  MyProject
			\fontdimen2\font=4pt
		\end{tcolorbox}		
		\item \textbf{-t}: Listing the contents of a tar archive.
		\newline
		Eg: List the content of tar archive:
		\begin{tcolorbox}[breakable,notitle,boxrule=-0pt,colback=black,colframe=black]
			\color{green}
			\fontdimen2\font=1em
			\$ tar -tvf MyProject.tar
			\fontdimen2\font=4pt
		\end{tcolorbox}		
		Eg: List the content of gzip compressed tar archive:
		\begin{tcolorbox}[breakable,notitle,boxrule=-0pt,colback=black,colframe=black]
			\color{green}
			\fontdimen2\font=1em
			\$ tar -tzvf MyProject.tar.gz
			\fontdimen2\font=4pt
		\end{tcolorbox}		
		Eg: List the content of bzip2 compressed tar archive:
		\begin{tcolorbox}[breakable,notitle,boxrule=-0pt,colback=black,colframe=black]
			\color{green}
			\fontdimen2\font=1em
			\$ tar -tjvf MyProject.tar.bz2
			\fontdimen2\font=4pt
		\end{tcolorbox}		
		\item \textbf{-x}: Extract the contents of a tar archive.
		\newline
		Eg: To extract a tar archive:
		\begin{tcolorbox}[breakable,notitle,boxrule=-0pt,colback=black,colframe=black]
			\color{green}
			\fontdimen2\font=1em
			\$ tar -xvf MyProject.tar
			\fontdimen2\font=4pt
		\end{tcolorbox}		
		Eg: To extract a gzip compressed tar archive:
		\begin{tcolorbox}[breakable,notitle,boxrule=-0pt,colback=black,colframe=black]
			\color{green}
			\fontdimen2\font=1em
			\$ tar -xzvf MyProject.tar.gz
			\fontdimen2\font=4pt
		\end{tcolorbox}		
		Eg: To extract a bzip2 compressed tar archive:
		\begin{tcolorbox}[breakable,notitle,boxrule=-0pt,colback=black,colframe=black]
			\color{green}
			\fontdimen2\font=1em
			\$ tar -xjvf MyProject.tar.bz2
			\fontdimen2\font=4pt
		\end{tcolorbox}		
	\end{enumerate}
	\bigskip
	Creating an archive in a different directory:
		Eg:
	\begin{tcolorbox}[breakable,notitle,boxrule=-0pt,colback=black,colframe=black]
		\color{green}
		\fontdimen2\font=1em
		\$ tar -cjvf /tmp/MyProject.tar.bz2  .
		\fontdimen2\font=4pt
	\end{tcolorbox}		
	
	
	
		
		
		
	\end{itemize} 
	
\end{flushleft}

\newpage

