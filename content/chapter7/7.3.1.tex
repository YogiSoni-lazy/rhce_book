\setlength{\columnsep}{3pt}
\begin{flushleft}
	Let's see some commands to archives and compress file or directory.
	\begin{itemize}
		\item \textbf{zip \& unzip}: 
		\begin{enumerate}
			\item \textbf{zip}: Compresses file/directory. It is available on Linux, Windows, and Mac OS. 
			\bigskip
			\begin{tcolorbox}[breakable,notitle,boxrule=0pt,colback=pink,colframe=pink]
				\color{black}
				\fontdimen2\font=1em
				Syntax: zip zip\_file\_name *
				\fontdimen2\font=4pt
			\end{tcolorbox}
			Eg:
			\begin{tcolorbox}[breakable,notitle,boxrule=-0pt,colback=black,colframe=black]
				\color{green}
				\fontdimen2\font=1em
				\# mkdir testiing
				\newline
				\# cd testing
				\newline
				\# touch \{1..10\} 
				\newline
				\# mkdir -p a/b/c
				\newline
				\newline
				\color{yellow}
				\# To compress \textbf{testing} folder, execute \textbf{zip} command inside the folder:
				\newline
				\color{green}
				\# zip data *
				\newline
				\newline
				\color{yellow}
				\# Confirm the creation of \textbf{data.zip} file
				\newline
				\color{green}
				\# ls
				\color{white}
				\newline
				1  10  2  3  4  5  6  7  8  9  a  data.zip
				\fontdimen2\font=4pt
			\end{tcolorbox}
			
			\bigskip
			\begin{tcolorbox}[breakable,notitle,boxrule=0pt,colback=yellow,colframe=yellow]
				\color{black}
				Note: zip command creates \textbf{data.zip} file automatically.
			\end{tcolorbox}
			
			Options with \textbf{zip} command:
			\begin{itemize}
				\item \textbf{-r}: To zip up an entire directory (including all subdirectories):
				\bigskip
				\begin{tcolorbox}[breakable,notitle,boxrule=0pt,colback=pink,colframe=pink]
					\color{black}
					\fontdimen2\font=1em
					Syntax: zip -r zip\_file\_name *
					\fontdimen2\font=4pt
				\end{tcolorbox}
				Eg:
				\begin{tcolorbox}[breakable,notitle,boxrule=-0pt,colback=black,colframe=black]
					\color{green}
					\fontdimen2\font=1em
					\# zip -r data *
					\fontdimen2\font=4pt
				\end{tcolorbox}
				
			\end{itemize}
			
			\bigskip\bigskip
			\item \textbf{unzip}: Used to decompress files/directories compressed using zip utility.
			\bigskip
			\begin{tcolorbox}[breakable,notitle,boxrule=0pt,colback=pink,colframe=pink]
				\color{black}
				\fontdimen2\font=1em
				Syntax: unzip argument
				\fontdimen2\font=4pt
			\end{tcolorbox}
			Eg:
			\bigskip
			\begin{tcolorbox}[breakable,notitle,boxrule=-0pt,colback=black,colframe=black]
				\color{green}
				\fontdimen2\font=1em
				\# unzip data.zip
				\fontdimen2\font=4pt
			\end{tcolorbox}
			
		\end{enumerate}
		\newpage
		\item \textbf{gzip} \& \textbf{gunzip}:
		\begin{enumerate}
			\item \textbf{gzip}: \textbf{Can compress only files.} Available on Linux and Mac OS. It gives compressed files \textbf{".gz"} extension.
			\bigskip
			\begin{tcolorbox}[breakable,notitle,boxrule=0pt,colback=pink,colframe=pink]
				\color{black}
				\fontdimen2\font=1em
				Syntax: gzip filename
				\fontdimen2\font=4pt
			\end{tcolorbox}
			Eg:
			\begin{tcolorbox}[breakable,notitle,boxrule=-0pt,colback=black,colframe=black]
				\color{green}
				\fontdimen2\font=1em
				\# gzip file.txt
				\fontdimen2\font=4pt
			\end{tcolorbox}
			
			\bigskip
			\begin{tcolorbox}[breakable,notitle,boxrule=0pt,colback=yellow,colframe=yellow]
				\color{black}
				Note: This command replaces \textbf{file.txt} with \textbf{file.txt.gz}
			\end{tcolorbox}
			
			
			\item \textbf{gunzip}: Uncompress the files with \textbf{".gz"} extension.
			\bigskip
			\begin{tcolorbox}[breakable,notitle,boxrule=0pt,colback=pink,colframe=pink]
				\color{black}
				\fontdimen2\font=1em
				Syntax: gunzip filename.gz
				\fontdimen2\font=4pt
			\end{tcolorbox}
			Eg:
			\begin{tcolorbox}[breakable,notitle,boxrule=-0pt,colback=black,colframe=black]
				\color{green}
				\fontdimen2\font=1em
				\# gunzip file.txt.gz
				\fontdimen2\font=4pt
			\end{tcolorbox}		
		\end{enumerate}
		
		
		
		\item \textbf{bzip2} \& \textbf{bunzip2}:
		\begin{enumerate}
			\item \textbf{bzip2}: \textbf{Can compress only files.} Available on Linux and Mac OS.  It gives compressed files \textbf{".bz2"} extension.
			\bigskip
			\begin{tcolorbox}[breakable,notitle,boxrule=0pt,colback=pink,colframe=pink]
				\color{black}
				\fontdimen2\font=1em
				Syntax: bzip2 filename
				\fontdimen2\font=4pt
			\end{tcolorbox}
			Eg:
			\begin{tcolorbox}[breakable,notitle,boxrule=-0pt,colback=black,colframe=black]
				\color{green}
				\fontdimen2\font=1em
				\# bzip2 file.txt
				\fontdimen2\font=4pt
			\end{tcolorbox}
			
			\bigskip
			\begin{tcolorbox}[breakable,notitle,boxrule=0pt,colback=yellow,colframe=yellow]
				\color{black}
				Note: This command replaces \textbf{file.txt} with \textbf{file.txt.bz2}
			\end{tcolorbox}
			
			
			\item \textbf{bunzip2}: Uncompress the files with \textbf{".bz2"} extension.
			\bigskip
			\begin{tcolorbox}[breakable,notitle,boxrule=0pt,colback=pink,colframe=pink]
				\color{black}
				\fontdimen2\font=1em
				Syntax: bunzip2 filename.bz2
				\fontdimen2\font=4pt
			\end{tcolorbox}
			Eg:
			\begin{tcolorbox}[breakable,notitle,boxrule=-0pt,colback=black,colframe=black]
				\color{green}
				\fontdimen2\font=1em
				\# bunzip2 file.txt.bz2
				\fontdimen2\font=4pt
			\end{tcolorbox}		
		\end{enumerate}		
		
		\newpage
		
		\item \textbf{tar}: Stands for "tape archive". \textbf{Used to create an archive of a directory.}
		\begin{tcolorbox}[breakable,notitle,boxrule=0pt,colback=pink,colframe=pink]
			\color{black}
			\fontdimen2\font=1em
			Syntax: tar [options] archive-file-name.tar directory\_name
			\fontdimen2\font=4pt
		\end{tcolorbox}
		Options with \textbf{tar} command:
		\begin{enumerate}
			\item \textbf{-c}: Create archive
			\newline
			\textbf{-v}: Stands for verbose. Tells tar to print all the filenames added to archive.
			\newline
			\textbf{-f}: The name of the archive
			Eg:
			\begin{tcolorbox}[breakable,notitle,boxrule=-0pt,colback=black,colframe=black]
				\color{green}
				\fontdimen2\font=1em
				\# tar -cvf MyProject.tar  MyProject
				\fontdimen2\font=4pt
			\end{tcolorbox}		
			\bigskip
			\begin{tcolorbox}[breakable,notitle,boxrule=0pt,colback=yellow,colframe=yellow]
				\color{black}
				Note: The \textbf{-cvf} will only archive the directory, but not compress it. To compress it further, you will need to use \textbf{"-z or -j"} options as shown below.
			\end{tcolorbox}
			\item \textbf{-z}: Compress a tar archive with the gzip command. You need to add extension \textbf{".tar.gz"}
			\newline
			Eg:
			\begin{tcolorbox}[breakable,notitle,boxrule=-0pt,colback=black,colframe=black]
				\color{green}
				\fontdimen2\font=1em
				\# tar -cvzf MyProject.tar.gz  MyProject
				\fontdimen2\font=4pt
			\end{tcolorbox}		
			\item \textbf{-j}: Compress a tar archive with bzip2 command. You need to add extension \textbf{".tar.bz2"}.
			\newline
			Eg:
			\begin{tcolorbox}[breakable,notitle,boxrule=-0pt,colback=black,colframe=black]
				\color{green}
				\fontdimen2\font=1em
				\# tar -cjvf MyProject.tar.bz2  MyProject
				\fontdimen2\font=4pt
			\end{tcolorbox}		
		
			To create an archive in a different directory:
			\begin{tcolorbox}[breakable,notitle,boxrule=-0pt,colback=black,colframe=black]
				\color{green}
				\fontdimen2\font=1em
				\$ tar -cjvf /tmp/MyProject.tar.bz2  MyProject
				\fontdimen2\font=4pt
			\end{tcolorbox}		
		
			\item \textbf{-t}: List the contents of a tar file.
			\newline
			Eg: 
			\begin{tcolorbox}[breakable,notitle,boxrule=-0pt,colback=black,colframe=black]
				\color{green}
				\fontdimen2\font=1em
				\# tar -tvf MyProject.tar
				\newline
				\# tar -tzvf MyProject.tar.gz
				\newline
				\#  tar -tjvf MyProject.tar.bz2
				\fontdimen2\font=4pt
			\end{tcolorbox}		

			\item \textbf{-x}: Extract the contents of a tar archive.
			\newline
			Eg:
			\begin{tcolorbox}[breakable,notitle,boxrule=-0pt,colback=black,colframe=black]
				\color{green}
				\fontdimen2\font=1em
				\# tar -xvf MyProject.tar
				\newline
				\# tar -xzvf MyProject.tar.gz
				\newline
				\# tar -xjvf MyProject.tar.bz2
				\fontdimen2\font=4pt
			\end{tcolorbox}		
		\end{enumerate}
		\bigskip

		
	\end{itemize} 
	
\end{flushleft}

\newpage

