
\begin{flushleft}
	
	\begin{itemize}
		\item Display all rich rules for the specified zone, or the default zone if no zone is specified.
		\bigskip
		\begin{tcolorbox}[breakable,notitle,boxrule=0pt,colback=pink,colframe=pink]
			\color{black}
			\fontdimen2\font=1em
			Syntax: 
			\newline
			firewall-cmd --list-rich-rules
			\fontdimen2\font=4pt
		\end{tcolorbox}

		\bigskip
		\bigskip
		\item Add <RICH RULE> to the specified zone, or the default zone if no zone is specified.
		\bigskip
		\begin{tcolorbox}[breakable,notitle,boxrule=0pt,colback=pink,colframe=pink]
			\color{black}
			\fontdimen2\font=1em
			Syntax: 
			\newline
			firewall-cmd --permanent --zone=<ZONE> --add-rich-rule='<RULE>'
			\newline
			firewall-cmd --reload
			\fontdimen2\font=4pt
		\end{tcolorbox}
	
		\bigskip
		\bigskip
		\item Remove <RICH RULE> to the specified zone, or the default zone if no zone is specified.
		\bigskip
		\begin{tcolorbox}[breakable,notitle,boxrule=0pt,colback=pink,colframe=pink]
			\color{black}
			\fontdimen2\font=1em
			Syntax: 
			\newline
			firewall-cmd --permanent --zone=<ZONE> --remove-rich-rule='<RULE>'
			\newline
			firewall-cmd --reload
			\fontdimen2\font=4pt
		\end{tcolorbox}
	
	\newpage
	
	\paragraph{Examples:}
	\bigskip
	\begin{enumerate}
		\item Reject all traffic from a "192.168.0.11/32" IP address in default zone:
		\newline
		Eg:
		\begin{tcolorbox}[breakable,notitle,boxrule=-0pt,colback=black,colframe=black]
			\color{green}
			\fontdimen2\font=1em
			\# firewall-cmd --permanent --add-rich-rule='rule family=ipv4 source address=192.168.0.11/32 reject'
			\newline
			\newline
			\# firewall-cmd --reload
			\fontdimen2\font=4pt
		\end{tcolorbox}
		
		\begin{figure}[h!]
			\centering
			\includegraphics[scale=.3]{content/chapter2/images/zones10.png}
			\caption{Sample output}
			\label{fig:command_prompt7}
		\end{figure}
		
		\newpage
		
		\item Allows port 8080 for a specific IP address "192.168.0.11/32":
		\newline
		Eg:
		\begin{tcolorbox}[breakable,notitle,boxrule=-0pt,colback=black,colframe=black]
			\color{green}
			\fontdimen2\font=1em
			\# firewall-cmd --add-rich-rule='rule family="ipv4" source address="192.168.0.11" port port=8080 protocol=tcp accept'
			\newline
			\newline
			\# firewall-cmd --reload
			\fontdimen2\font=4pt
		\end{tcolorbox}
		
		\begin{figure}[h!]
			\centering
			\includegraphics[scale=.3]{content/chapter2/images/zones11.png}
			\caption{Sample output}
			\label{fig:command_prompt9}
		\end{figure}
			
		\newpage
		\item Accept all TCP packets on ports 7900, up to and including port 7905, in the vnc zone for the 192.168.1.0/24 subnet.
		\newline
		Eg:
		\begin{tcolorbox}[breakable,notitle,boxrule=-0pt,colback=black,colframe=black]
			\color{green}
			\fontdimen2\font=1em
			\# firewall-cmd --permanent --zone=vnc --add-rich-rule='rule family=ipv4 source address=192.168.1.0/24 port port=7900-7905 protocol=tcp accept'
			\newline
			\newline
			\# firewall-cmd --reload
			\fontdimen2\font=4pt
		\end{tcolorbox}
		
		\begin{figure}[h!]
			\centering
			\includegraphics[scale=.3]{content/chapter2/images/zones12.png}
			\caption{Sample output}
			\label{fig:command_prompt10}
		\end{figure}
		
	\end{enumerate}

	\end{itemize}
	
\end{flushleft}

\newpage

