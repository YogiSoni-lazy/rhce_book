\setlength{\columnsep}{5pt}
\begin{flushleft}
	\paragraph{}
	\begin{itemize}
		\item A Linux distribution (or distro) is made from the \textbf{Linux kernel and collection of software}.
		\item Almost one thousand Linux distributions exist.
		\item Free and community managed distributions are:
		\begin{itemize}
			\item Fedora Linux 
			\item openSUSE
			\item Ubuntu
			\item Debian
			\item Slackware
			\item Gentoo 
			\item Arch Linux
		\end{itemize}
		\item Popular commercially backed distributions are:
		\begin{itemize}
			\item Red Hat Enterprise Linux
			\item SUSE
			\item Oracle Linux
		\end{itemize}
	\end{itemize}
	\paragraph{What is upstream and downstream?}
	\begin{itemize}
		\item The term 'upstream' refers to the \textbf{original version of a software}.
		\item Downstream is the \textbf{refined product code} based on original software version.
		\item Eg:
		\begin{itemize}
			\item \textbf{Fedora} is the upstream to \textbf{Red Hat Enterprise Linux (RHEL)}.
			\item \textbf{Debian} is the upstream to \textbf{Ubuntu}.
		\end{itemize}
	\end{itemize}
\end{flushleft}
\newpage
